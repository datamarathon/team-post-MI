\documentclass{tufte-handout}
\title{MIMIC-II Intro for MAVERIC NLP Group}
\author{Andy Zimolzak, MD, MMSc}
\date{March 3, 2015}

\begin{document}

\maketitle

~\\

The Multiparameter Intelligent Monitoring in Intensive Care II
(MIMIC-II) database consists of clinical data on about 32,000
intensive care unit (ICU) admissions at Beth Israel Deaconess Medical
Center from 2001--2008.\footnote{Saeed M, Villarroel M, Reisner AT,
  Clifford G, Lehman L-W, Moody G, \emph{et al.} Multiparameter
  Intelligent Monitoring in Intensive Care II: a public-access
  intensive care unit database. Crit.\ Care Med. 2011
  May;39(5):952–960. PMID: 21283005}\footnote{Lee J, Scott DJ,
  Villarroel M, Clifford GD, Saeed M, Mark RG. Open-access MIMIC-II
  database for intensive care research. Conf Proc IEEE Eng Med Biol
  Soc 2011;2011:8315–8318. PMID: 22256274} It is maintained by MIT's
Laboratory of Computational Physiology. Data includes demographics,
vital signs, medications, laboratory results,
deidentified\footnote{Neamatullah I, Douglass MM, Lehman L-WH, Reisner
  A, Villarroel M, Long WJ, \emph{et al.} Automated de-identification
  of free-text medical records. BMC Med Inform Decis Mak 2008;8:32.
  PMID: 18652655 } clinical notes (ICU physician, nurse, radiology),
and 125 Hz physiological waveforms. MIMIC-II contains data \emph{only}
about patients who spent some time in an ICU. If the patient was
transferred into the ICU from the ``general wards,'' then MIMIC went
back in time to grab structured data from the wards (usually labs but
never vital signs or notes). MIMIC-II also has Social Security Death
Index data, so it can tell you when/if a patient died \emph{after}
leaving the hospital.

Note that dates are also deidentified by shifting them \emph{a lot,}
such as into the year 3300. If your analysis software can't comprehend
this, you could subtract 1000 from dates in SQL before exporting.
Dates do preserve seasons of the year. 


\section{Permissions and logistics of setup}

You can get structured data (no text notes) on 4000 deceased patients
without signing anything at all. Waveforms also do not require data
use agreement (DUA). How to get full access, in four easy steps:
Complete any online human research course. Make a PhysioNetWorks
account. Apply for access to MIMIC II Clinical by clicking ``I Agree''
under the MIMIC section of PhysioNetWorks. Reply to the e-mail you
get, complete the DUA, and attach your human subject certificate. The
DUA is 243 words long, which is \emph{extraordinarily} simple. A key
point, though, is provision number 3: ``I will not share access to
restricted data from PhysioNet with anyone else.''

The MIT researchers\footnote{Scott DJ, Lee J, Silva I, Park S, Moody
  GB, Celi LA, \emph{et al.} Accessing the public MIMIC-II intensive
  care relational database for clinical research. BMC Med Inform Decis
  Mak 2013;13:9. PMID: 23302652 } package MIMIC in several ways,
including a Web client and a virtual machine. I \emph{highly}
recommend both.

\paragraph{Virtual machine:} This has Postgres on Ubuntu Linux on Oracle
VirtualBox. The MIMIC-II data except for waveforms is about 55 GB (VM
by default will occupy about 80 GB for its disk image). It took me
1--2 days to set up the VM: mostly troubleshooting VirtualBox,
particularly its network setup. The VM comes with a script that
downloads the flat files and does all the indexing. Troubleshooting
VirtualBox for me is easier than troubleshooting Postgres, but people
with more database skills than me might prefer to run their favorite
SQL directly rather than Postgres on top of the VM.

\paragraph{Query builder:} This is a Web gateway that lets you enter
SQL.\footnote{https://mimic2app.csail.mit.edu/ querybuilder/} Log in
with your PhysioNet username and password.

\section{Actually using MIMIC-II}

If you just want text notes, without any corroborating structured
data, the one and only table for you is \texttt{NOTEEVENTS}. Fun
query: \texttt{select text from noteevents where ROWNUM <= 10}. Or
also try (be patient): \texttt{select count(*), category from
  noteevents group by category}. Results below:

~\\

\begin{tabular}{l r}
\hline
Category & Count \\
\hline
Nursing/Other & 800,070 \\
RADIOLOGY\_REPORT & 384,513 \\
DISCHARGE\_SUMMARY & 31,958 \\
MD Notes & 23,285 \\
\hline
\end{tabular}

~\\

There are 40 tables and 5 premade views. It is reasonably well
documented. There is a user
guide\footnote{mimic.physionet.org/\\Ecocide/UserGuide.pdf} that has
E-R diagrams of the schema, etc., and there is a SQL
cookbook.\footnote{www.physionet.org/\\mimic2/demo/\\MIMICIICookBook\_v1.pdf}
Additionally, I've participated in several hackathons using MIMIC-II
as a data source, one of which encouraged people to use
GitHub.\footnote{https://github.com/datamarathon} This can be a
secondary resource of queries by example.

Query builder uses Oracle-like syntax. So you should say:
\texttt{select * from d\_codeditems where ROWNUM <= 10}. That is, it
doesn't seem to understand \texttt{SELECT TOP 10} or \texttt{LIMIT
  10}. It also hates semicolons. The MIMIC VM is Postgres, so there
you would use \texttt{LIMIT 10}. 

Other good tables: chartevents, icd9, ioevents, labevents, medevents, deliveries.
Chartevents has \emph{lots} of vital sign and other data. Note that it
is ``charted'' by actual humans, which is arguably a benefit because
they vet it before charting. There is an important relation between
chartevents.itemid and d\_chartitems.itemid. This relation exists
analogously for many of the $x$-events and d\_$x$-items tables. There
is also a useful view comorbidity\_scores, which contains scores on
how many other significant diseases a patient has in the past.
Finally, icustay\_detail is an \emph{extremely} useful view with 42
columns, somewhat denormalized. The icustay\_detail view includes SAPS
and SOFA, which are scores of ``how sick'' a patient is at a given
point in time.

Some important ``unit of analysis'' ID numbers are icustay\_id,
hadm\_id, and subject\_id. Remember, ICU stay $\subset$ hospital
admission $\subset$ subject. 

\end{document}

% LocalWords:  Multiparameter Saeed Villarroel Reisner Crit RG Conf Proc IEEE
% LocalWords:  Neamatullah WH WJ de BMC Decis Mak waveforms DUA PhysioNet Celi
% LocalWords:  Postgres VirtualBox username hackathons PhysioNetWorks ROWNUM
% LocalWords:  querybuilder NOTEEVENTS noteevents premade MIMICIICookBook pdf
% LocalWords:  codeditems chartevents icd ioevents medevents itemid chartitems
% LocalWords:  icustay hadm
